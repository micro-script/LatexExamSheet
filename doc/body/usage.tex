% !TeX root = ../exam-zh.tex

\section{使用说明}

\subsection{基本用法}

以下是一份简单的 \TeX{} 文档,它演示了 \cls{exam-zh} 的最基本用法:

\begin{latexexample}[deletetexcs={\documentclass},
    moretexcs={\chapter},morekeywords={\documentclass},
    emph={[2]document}]
  % main.tex
  \documentclass{exam-zh}
  \begin{document}
    \section{Welcome to exam-zh!}
    你好,\LaTeX{}!
  \end{document}
\end{latexexample}


按照~\ref{subsec:编译方式} 小节中的方式编译,您应当得到一篇 1 页的文档。


\subsection{编译方式} \label{subsec:编译方式}

本模板不支持 \pdfTeX{} 引擎,仅支持使用 \XeLaTeX{} 。为了生成正确的目录、脚注以及交叉引用,您至少需要连续编译两次。

以下代码中,假设您的 \TeX{} 源文件名为 \file{example.tex}。请在命令行中执行
\begin{shellexample}[morekeywords={xelatex}]
  xelatex example
\end{shellexample}


\subsection{模板选项}

所谓“模板选项”,指需要在引入文档类的时候指定的选项:
\begin{latexexample}[deletetexcs={\documentclass},
    morekeywords={\documentclass}]
  \documentclass(*\oarg{模板选项}*){exam-zh}
\end{latexexample}

有些模板选项为布尔型,它们只能在 \opt{true} 和 \opt{false}
中取值。对于这些选项,\kvopt{\meta{选项}}{true} 中的“|= true|”
可以省略。

\cls{exam-zh} 的模版选项接口与 \cls{ctexart} 相同,具体可 \cmd{texdoc ctex} 查阅 \pkg{ctex} 宏包文档。



\subsection{参数设置}

\begin{function}{\examsetup}
  \begin{ccnusyntax}[morekeywords={\examsetup}]
    \examsetup(*\marg{键值列表}*)
  \end{ccnusyntax}
  本模板提供了一系列选项,可由您自行配置。载入文档类之后,以下所有选项均可通过统一的命令 \cs{examsetup} 来设置。
\end{function}

\cs{examsetup} 的参数是一组由(英文)逗号隔开的选项列表,列表中的
选项通常是 \kvopt{\meta{key}}{\meta{value}} 的形式。部分选项的
\meta{value} 可以省略。对于同一项,后面的设置将会覆盖前面的设置。
在下文的说明中,将用\textbf{粗体}表示默认值。

\cs{examsetup} 采用 \LaTeX3 风格的键值设置,支持不同类型以及多种
层次的选项设定。键值列表中,“|=|”左右的空格不影响设置;但需注意,
参数列表中\emph{不可以出现空行}。

与模板选项相同,布尔型的参数可以省略 \kvopt{\meta{选项}}{true}
中的“\kvopt{}{true}”。

另有一些选项包含子选项,如 \opt{page} 和 \opt{choices} 等。它们可以
按如下两种等价方式来设定:

\begin{latexexample}[morekeywords={\examsetup},
    emph={[1]page,size,choices,column-sep,label-pos,label-sep,max-columns}]
  \examsetup{
    page = {
      size = a3paper
    },
    choices = {
      column-sep  = 1em,
      label-pos   = auto,
      label-sep   = 0.5em,
      max-columns = 4
    }
  }
\end{latexexample}

或者

\begin{latexexample}[morekeywords={\examsetup},
    emph={[1]page,size,choices,column-sep,label-pos,label-sep,max-columns}]
  \examsetup{
    page/size            = a3paper,
    choices/column-sep   = 1em,
    choices/label-pos    = auto,
    choices/label-sep    = 0.5em,
    choices/max-columns  = 4
  }
\end{latexexample}


注意 “|/|” 的前后均不可以出现空白字符。


\subsubsection{页面设置} \label{subsubsec:页面设置}

\begin{function}{page}
  \begin{ccnusyntax}[emph={[1]page}]
    page = (*\marg{键值列表}*)
    page/(*\meta{key}*) = (*\meta{value}*)
  \end{ccnusyntax}
  该选项包含许多子项目,用于设置页面设置。具体内容见下。
\end{function}


\begin{function}{style/font}
  \begin{ccnusyntax}[emph={[1]font}]
    font = (*<newtx|(times)|stixtwo|xits|tg|none>*)
  \end{ccnusyntax}
  【本|硕|博】设置西文字体(包括数学字体)。具体配置见表~\ref{tab:font}。
\end{function}

\begin{table}[htbp]
  \centering
  \begin{threeparttable}
    \caption{西文字体配置}
    \label{tab:font}
    \small
    \begin{tabular}{ccccc}
      \toprule
        & \textbf{正文字体} & \textbf{无衬线字体} & \textbf{等宽字体} & \textbf{数学字体} \\
      \midrule
      |stixtwo| & STIX Two Text   & TG Heros\tnote{a} & TG Cursor   & STIX Two Math \\
      |xits |   & XITS            & TG Heros & TG Cursor   & XITS Math \\
      |times|\tnote{b}   & Times New Roman & Arial    & Courier New & newtxmath \\
      |newtx|   & TG Termes   & TG Heros & TG Cursor   & newtxmath \\
      |tg|      & TG Termes       & TG Heros & TG Cursor   & TG Termes Math \\
      \bottomrule
    \end{tabular}
    \begin{tablenotes}
      % \item[a] “LM”是 Latin Modern 的缩写。
      \item[a] “TG”是 TeX Gyre 的缩写。
      \item[b] 本行中,Times New Roman、Arial 和 Courier New 是商业字体,
        不包含在 \TeXLive{} 发行版中,但在 Windows 和 macOS 系统上均默认安装。
    \end{tablenotes}
  \end{threeparttable}
  \end{table}

\subsubsection{密封线} \label{subsubsec:密封线}

\begin{function}{sealline}
  \begin{ccnusyntax}[emph={[1]sealline}]
    sealline = (*\marg{键值列表}*)
    sealline/(*\meta{key}*) = (*\meta{value}*)
  \end{ccnusyntax}
  该选项包含许多子项目,用于设置密封线。具体内容见下。
\end{function}


\subsubsection{字体} \label{subsubsec:字体}

\begin{function}{font}
  \begin{ccnusyntax}[emph={[1]font}]
    font = (*\marg{键值列表}*)
  \end{ccnusyntax}
  设置西文字体。
\end{function}

\begin{function}{math-font}
  \begin{ccnusyntax}[emph={[1]math-font}]
    math-font = (*\marg{键值列表}*)
  \end{ccnusyntax}
  设置数学字体。
\end{function}


\subsubsection{题干} \label{subsubsec:题干}

\begin{function}{question}
  \begin{ccnusyntax}[emph={[1]question}]
    question = (*\marg{键值列表}*)
    question/(*\meta{key}*) = (*\meta{value}*)
  \end{ccnusyntax}
  该选项包含许多子项目,用于设置题干。具体内容见下。
\end{function}


\subsubsection{选择题} \label{subsubsec:选择题}

\begin{function}{choices}
  \begin{ccnusyntax}[emph={[1]choices}]
    choices = (*\marg{键值列表}*)
    choices/(*\meta{key}*) = (*\meta{value}*)
  \end{ccnusyntax}
  该选项包含许多子项目,用于设置页面设置。具体内容见下。
\end{function}


\subsubsection{填空题} \label{subsubsec:填空题}

\begin{function}{fillin}
  \begin{ccnusyntax}[emph={[1]fillin}]
    fillin = (*\marg{键值列表}*)
    fillin/(*\meta{key}*) = (*\meta{value}*)
  \end{ccnusyntax}
  该选项包含许多子项目,用于设置页面设置。具体内容见下。
\end{function}