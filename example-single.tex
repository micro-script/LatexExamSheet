\documentclass{exam-zh-pro}

\examsetup{
  page/size=a4paper,
  solution = {
    % show-solution = hide,         % 隐藏答案
    % show-solution = show-stay,    % 原处显示答案
    show-solution = show-move     % 答案移动到最后一页
  }
}


\title{2021 年普通高等学校招生全国统一考试}

\subject{数学(理科)}



\begin{document}

\section{%
  选择题:本题共 8 小题,每小题 5 分,共 40 分。
  在每小题给出的四个选项中,只有一项是符合题目要求的。
}


\begin{question}[points = 2]
  设集合 $A = \{x \mid -1 < x < 4\}$,$B = \{2, 3, 4, 5\}$,则 $A \cap B = $ \paren[A]

  \begin{choices}
    \item $\{2\}$
    \item $\{2, 3\}$
    \item $\{3, 4\}$
    \item $\{2, 3, 4\}$
  \end{choices}
\end{question}

\begin{solution}
  略
\end{solution}

% \begin{question}
%   已知 $z = 2 - \iu$,则 $z (\bar{z} + \iu) = $ \paren
%   \begin{choices}
%     \item $6 - 2\iu$
%     \item $2 - 2\iu$
%     \item $6 + 2\iu$
%     \item $4 + 2\iu$
%   \end{choices}
% \end{question}



% \section{填空题:本题共 4 小题,每小题 5 分,共 20 分。}


% \begin{question}
%   已知函数 $f(x) = x^3 (a \cdot 2^x - 2^{-x})$ 是偶函数,则 $a = $ \fillin[$1$] 。
% \end{question}


% \begin{question}
%   已知 $O$ 为坐标原点,抛物线 $C \colon y^2 = 2px$($p > 0$)的焦点为 $F$,
%   $P$ 为 $C$ 上一点,$PF$ 与 $x$ 轴垂直,$Q$ 为 $x$ 轴上一点,且 $PQ \perp OP$,
%   若 $|FQ| = 6$,则 $C$ 的准线方程为 \fillin[$\dfrac{1}{3}$] 。
% \end{question}



% \section{解答题:本题共 6 小题,共 70 分。解答应写出文字说明、证明过程或者演算步骤。}


% \begin{problem}[points = 10]
%   已知数列 $\{a_n\}$ 满足 $a_1 = 1$,$a_{n+1} =
%     \begin{cases}
%       a_n + 1, \quad \text{$n$ 为奇数,} \\
%       a_n + 2, \quad \text{$n$ 为偶数。}
%     \end{cases}$
%   \begin{enumerate}
%     \item 记 $b_n = a_{2n}$,写出 $b_1$,$b_2$,并求数列 $\{b_n\}$ 的通项公式;
%     \item 求 $\{a_n\}$ 的前 $20$ 项和。
%   \end{enumerate}
% \end{problem}

% \begin{solution}
%   函数的定义域为 $(0, +\infty)$,
%   又 \[f^{\prime}(x) = 1 - \ln x-1 = -\ln x, \score{2}\]
%   当 $x \in(0, 1)$ 时, $f^{\prime}(x) > 0$, 当 $x \in(1, +\infty)$ 时, $f^{\prime}(x) < 0$,
%   故 $f(x)$ 的递增区间为 $(0,1)$, 递减区间为 $(1, +\infty)$.

%   函数的定义域为 $(0, +\infty)$,
%   又 \[f^{\prime}(x) = 1 - \ln x-1 = -\ln x, \score{2}\]
%   当 $x \in(0, 1)$ 时, $f^{\prime}(x) > 0$, 当 $x \in(1, +\infty)$ 时, $f^{\prime}(x) < 0$,
%   故 $f(x)$ 的递增区间为 $(0,1)$, 递减区间为 $(1, +\infty)$.
% \end{solution}




% \begin{problem}[points = 12]
%   记 $\triangle ABC$ 的内角 $A$,$B$,$C$ 的对边分别为 $a$,$b$,$c$。
%   已知 $b^2 = ac$,点 $D$ 在边 $AC$ 上,$BD \sin\angle ABC = a \sin C$。
%   \begin{enumerate}
%     \item 证明:$BD = b$;
%     \item 若 $AD = 2 DC$,求 $\cos\angle ABC$。
%   \end{enumerate}
% \end{problem}

% \begin{solution}
%   \begin{enumerate}
%     \item 函数的定义域为 $(0, +\infty)$,
%     又 \[f^{\prime}(x) = 1 - \ln x-1 = -\ln x, \score{2}\]
%     当 $x \in(0, 1)$ 时, $f^{\prime}(x) > 0$, 当 $x \in(1, +\infty)$ 时, $f^{\prime}(x) < 0$,
%     故 $f(x)$ 的递增区间为 $(0,1)$, 递减区间为 $(1, +\infty)$.
%     \item 函数的定义域为 $(0, +\infty)$,
%     又 \[f^{\prime}(x) = 1 - \ln x-1 = -\ln x, \score{2}\]
%     当 $x \in(0, 1)$ 时, $f^{\prime}(x) > 0$, 当 $x \in(1, +\infty)$ 时, $f^{\prime}(x) < 0$,
%     故 $f(x)$ 的递增区间为 $(0,1)$, 递减区间为 $(1, +\infty)$.
%   \end{enumerate}
% \end{solution}



\end{document}