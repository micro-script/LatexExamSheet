% !TeX encoding = UTF-8
% !TeX program = xelatex
% !TeX spellcheck = en_US


\documentclass{exam-zh}
\usepackage{siunitx}

\examsetup{
  page = {
    size            = a3paper,
    show-columnline = false
  },
  style/fullwidth-stop = catcode,
  fillin/no-answer-type = none,
  sealline = {
    show        = true,
    scope       = mod-2,
    circle-show = false,
    line-type   = solid,
    odd-info-content = {
      {\heiti \zihao{4}姓名} {\underline{\hspace*{8em}}},
      {\heiti \zihao{4}准考证号} {\examsquare{9}},
      {\heiti \zihao{4}考场号} {\examsquare{2}},
      {\heiti \zihao{4}座位号} {\examsquare{2}},
    },
    odd-info-xshift = 12mm,
    text = {此卷只装订不密封},
    text-width = 0.98\textheight,
    text-format  = \zihao{-3}\sffamily,
    text-xshift = 20mm
  },
  square = {
    x-length = 1.8em,
    y-length = 1.6em
  }
}


% 行内公式统一按照行间的样式
\everymath{\displaystyle}

\title{2021 年普通高等学校招生全国统一考试}

\subject{数学(理科)}


\begin{document}

\tableofcontents

\chapter{2021 年普通高等学校招生全国统一考试}

% \information{
%   姓名\underline{\hspace{6em}},
%   座位号\underline{\hspace{15em}}
% }
% \warning{(在此卷上答题无效)}

\secret

\maketitle

本试卷共 4 页,22 题。全卷满分 150 分。考试用时 120 分钟。

\goodluck

\begin{notice}
  \item 答题前,先将自己的姓名、准考证号、考场号、座位号填写在试卷和答题卡上,
    并将准考证号条形码粘贴在答题卡上的指定位置。
  \item 选择题的作答:每小题选出答案后,用 2B 铅笔把答题卡上对应题目的答案标号涂黑。
    写在试卷、草稿纸和答题卡上的非答题区域均无效。
  \item 填空题和解答题的作答:用黑色签字笔直接答在答题卡上对应的答题区域内。
    写在试卷、草稿纸和答题卡上的非答题区域均无效。
  \item 考试结束后,请将本试卷和答题卡一并上交。
\end{notice}



\section{%
  选择题:本题共 8 小题,每小题 5 分,共 40 分。
  在每小题给出的四个选项中,只有一项是符合题目要求的。
}

% 1.
\begin{question}[points = 2]
  设集合 $A = \{x \mid -1 < x < 4\}$,$B = \{2, 3, 4, 5\}$,则 $A \cap B = $ \paren[B]

  \begin{choices}
    \item $\{2\}$
    \item $\{2, 3\}$
    \item $\{3, 4\}$
    \item $\{2, 3, 4\}$
  \end{choices}
\end{question}

% 2.
\begin{question}
  已知 $z = 2 - \iu$,则 $z (\bar{z} + \iu) = $ \paren
  \begin{choices}
    \item $6 - 2\iu$
    \item $2 - 2\iu$
    \item $6 + 2\iu$
    \item $4 + 2\iu$
  \end{choices}
\end{question}

% 3.
\begin{question}
  已知圆锥的底面半径为 $\sqrt{2}$,其侧面展开图为一个半圆,则该圆锥的母线长为 \paren
  \begin{choices}
    \item $2$
    \item $2 \sqrt{2}$
    \item $4$
    \item $4 \sqrt{2}$
  \end{choices}
\end{question}

% 4.
\begin{question}
  下列区间中,函数 $f(x) = 7 \sin \left( x - \frac{\uppi}{6} \right)$ 的单调递增区间是 \paren
  \begin{choices}
    \item $\left( 0               , \frac{\uppi}{2}  \right)$
    \item $\left( \frac{\uppi}{2} , \uppi            \right)$
    \item $\left( \uppi           , \frac{3\uppi}{2} \right)$
    \item $\left( \frac{3\uppi}{2}, 2\uppi           \right)$
  \end{choices}
\end{question}

% 5.
\begin{question}
  已知 $F_1$,$F_2$ 是椭圆 $C \colon \frac{x^2}{9} + \frac{y^2}{4} = 1$ 的两个焦点,
  点 $M$ 在 $C$ 上,则 $|M F_1| \cdot |M F_2|$ 的最大值为 \paren
  \begin{choices}
    \item $13$
    \item $12$
    \item $9$
    \item $6$
  \end{choices}
\end{question}

% 6.
\begin{question}
  若 $\tan\theta = -2$,则 $\frac{\sin\theta (1 + \sin 2\theta)}{\sin\theta + \cos\theta} = $ \paren
  \begin{choices}
    \item $-\frac{6}{5}$
    \item $-\frac{2}{5}$
    \item $\frac{2}{5}$
    \item $\frac{6}{5}$
  \end{choices}
\end{question}

% 7.
\begin{question}
  若过点 $(a, b)$ 可作曲线 $y = \eu^x$ 的两条切线,则 \paren
  \begin{choices}
    \item $\eu^b < a$
    \item $\eu^a < b$
    \item $0 < a < \eu^b$
    \item $0 < b < \eu^a$
  \end{choices}
\end{question}

% 8.
\begin{question}
  有 $6$ 个相同的球,分别标有数字 $1$,$2$,$3$,$4$,$5$,$6$,从中有放回地随机取两次,每次取 $1$ 个球,
  甲表示事件“第一次去出的球的数字是 $1$”,
  乙表示事件“第二次取出的球的数字是 $2$”,
  丙表示事件“两次取出的球的数字之和是 $8$”,
  丁表示事件“两次取出的球的数字之和是 $7$”,则 \paren
  \begin{choices}
    \item 甲与丙相互独立
    \item 甲与丁相互独立
    \item 乙与丙相互独立
    \item 丙与丁相互独立
  \end{choices}
\end{question}



\section{%
  选择题:本题共 4 小题,每小题 5 分,共 20 分。
  在每小题给出的选项中,有多项符合题目要求的。
  全部选对的得 5 分,部分选择的得 2 分,有选错的得 0 分。
}

% 9.
\begin{question}
  有一组样本数据 $x_1, x_2, \dots, x_n$,由 这组数据的到新样本数据 $y_1, y_2, \dots, y_n$,
  其中 $y_i = x_i + c$($i = 1, 2, \dots, n$) 为非零常数,则 \paren
  \begin{choices}
    \item 两组样本数据的样本平均数相同
    \item 两组样本数据的样本中位数相同
    \item 两组样本数据的样本标准差相同
    \item 两组样本数据的样本极差相同
  \end{choices}
\end{question}

% 10.
\begin{question}
  已知 $O$ 为坐标原点,点
  $P_1(\cos\alpha,  \sin\alpha)$,
  $P_2(\cos\beta , -\sin\alpha)$,
  $P_3(\cos(\alpha + \beta),  \sin(\alpha + \beta))$,
  $A(1, 0)$ \paren
  \begin{choices}
    \item $|\overrightarrow{OP_1}| = |\overrightarrow{OP_2}|$
    \item $|\overrightarrow{AP_1}| = |\overrightarrow{AP_2}|$
    \item $\overrightarrow{OA} \cdot \overrightarrow{OP_3}
      = \overrightarrow{OP_1} \cdot \overrightarrow{OP_2}$
    \item $\overrightarrow{OA} \cdot \overrightarrow{OP_1}
      = \overrightarrow{OP_2} \cdot \overrightarrow{OP_3}$
  \end{choices}
\end{question}

% 11.
\begin{question}
  已知点 $P$ 在圆 $(x - 5)^2 + (y - 5)^2 = 16$ 上,点 $A(4, 0)$,$B(0, 2)$,则 \paren
  \begin{choices}
    \item 点 $P$ 到直线 $AB$ 的距离小于 $10$
    \item 点 $P$ 到直线 $AB$ 的距离大于 $2$
    \item 点 $\angle PBA$ 最小时,$|PB| = 3 \sqrt{2}$
    \item 点 $\angle PBA$ 最大时,$|PB| = 3 \sqrt{2}$
  \end{choices}
\end{question}

% 12.
\begin{question}
  在正三棱柱 $ABC$-$A_1 B_1 C_1$ 中,$AB = A A_1 = 1$,点 $P$ 满足
  $\overrightarrow{BP} = \lambda \overrightarrow{BC} + \mu \overrightarrow{BB_1}$,
  其中 $\lambda \in [0, 1]$,$\mu \in [0, 1]$,则 \paren
  \begin{choices}
    \item 当 $\lambda = 1$ 时,$\triangle A B_1 P$ 的周长为定值
    \item 当 $\mu = 1$ 时,三棱锥 $P$-$A_1 B C$ 的体积为定值
    \item 当 $\lambda = \frac{1}{2}$ 时,有且仅有一个点 $P$,使得 $A_1 P \perp BP$
    \item 当 $\mu = \frac{1}{2}$ 时,有且仅有一个点 $P$,使得 $A_1 B \perp \text{平面} A B_1 P$
  \end{choices}
\end{question}



\section{填空题:本题共 4 小题,每小题 5 分,共 20 分。}

% 13.
\begin{question}
  已知函数 $f(x) = x^3 (a \cdot 2^x - 2^{-x})$ 是偶函数,则 $a = $ \fillin[$1$] 。
\end{question}

% 14.
\begin{question}
  已知 $O$ 为坐标原点,抛物线 $C \colon y^2 = 2px$($p > 0$)的焦点为 $F$,
  $P$ 为 $C$ 上一点,$PF$ 与 $x$ 轴垂直,$Q$ 为 $x$ 轴上一点,且 $PQ \perp OP$,
  若 $|FQ| = 6$,则 $C$ 的准线方程为 \fillin[$\dfrac{1}{3}$] 。
\end{question}

% 15.
\begin{question}
  函数 $f(x) = |2x - 1| - 2 \ln x$ 的最小值为 \fillin[width = 4em][] 。
\end{question}

% 16.
\begin{question}
  某校学生在研究民间剪纸艺术时,发现剪纸时经常会沿纸的某条对称轴把纸对折。
  规格为 \qtyproduct{20 x 12}{dm} 的长方形纸,对折 $1$ 次共可以得到
  \qtyproduct{10 x 12}{dm}, \qtyproduct{20 x 6}{dm} 两种规格的图形,
  它们的面积之和 $S_1 = \qty{240}{dm^2}$,
  对折 $2$ 次共可以得到 \qtyproduct{5 x 12}{dm},\qtyproduct{10 x 6}{dm},
  \qtyproduct{20 x 3}{dm} 三种规格的图形,它们的面积之和 $S_2 = \qty{240}{dm^2}$,
  以此类推。则对折 $4$ 次共可以得到不同规格图形的种数为 \fillin ;
  如果对折 $n$ 次,那么 $\sum_{k=1}^n S_k = $ \fillin \unit{dm^2}。
\end{question}



\section{解答题:本题共 6 小题,共 70 分。解答应写出文字说明、证明过程或者演算步骤。}

% 17.
\begin{problem}[points = 10]
  已知数列 $\{a_n\}$ 满足 $a_1 = 1$,$a_{n+1} =
    \begin{cases}
      a_n + 1, \quad \text{$n$ 为奇数,} \\
      a_n + 2, \quad \text{$n$ 为偶数。}
    \end{cases}$
  \begin{enumerate}
    \item 记 $b_n = a_{2n}$,写出 $b_1$,$b_2$,并求数列 $\{b_n\}$ 的通项公式;
    \item 求 $\{a_n\}$ 的前 $20$ 项和。
  \end{enumerate}
\end{problem}

% 18.
\begin{problem}[points = 12]
  某学校组织“一带一路”知识竞赛,有 A,B 两类问题。
  每位参加比赛的同学现在两类问题中选择一类并从中随机抽取一个问题回答,
  若回答错误则该同学比赛结束;
  若回答正确则从另一类问题中再随机抽取一个问题回答,无论回答正确与否,该同学比赛结束。
  A 类问题中的每个问题回答正确的 $20$ 分,否则得 $0$ 分;
  B 类问题中的每个问题回答正确的 $80$ 分,否则得 $0$ 分。

  已知小明能正确回答 A 类问题的概率为 $0.8$,能正确回答 B 类问题的概率为 $0.6$,
  且能正确回答问题的概率与回答次序无关。
  \begin{enumerate}
    \item 若小明先回答 A 类问题,记 $X$ 为小明的累计得分,求 $X$ 的分布列;
    \item 为使累计得分的期望最大,小明应选择先回答哪类问题?并说明理由。
  \end{enumerate}
\end{problem}

% 19.
\begin{problem}[points = 12]
  记 $\triangle ABC$ 的内角 $A$,$B$,$C$ 的对边分别为 $a$,$b$,$c$。
  已知 $b^2 = ac$,点 $D$ 在边 $AC$ 上,$BD \sin\angle ABC = a \sin C$。
  \nopagebreak
  \begin{enumerate}
    \item 证明:$BD = b$;
    \item 若 $AD = 2 DC$,求 $\cos\angle ABC$。
  \end{enumerate}
\end{problem}

% 20.
\begin{problem}[points = 12]
  如图,在三棱锥 $A$-$BCD$ 中,$\text{平面} ABD \perp \text{平面} BCD$,
  $AB = AD$,$O$ 为 $BD$ 的重点。
  \begin{enumerate}
    \item 证明:$OA \perp CD$;
    \item 若 $\triangle OCD$ 是变长为 $1$ 的等边三角形,点 $E$ 在棱 $AD$ 上,
      $DE = 2 EA$,且二面角 $E$-$BC$-$D$ 的大小为 \ang{45},
      求三棱锥 $A$-$BCD$ 的体积。
  \end{enumerate}
\end{problem}

% 21.
\begin{problem}[points = 12]
  在平面直角坐标系 $xOy$ 中,已知点 $F_1 (-\sqrt{17}, 0)$,$F_2 (\sqrt{17}, 0)$,
  点 $M$ 满足 $|M F_1| - |M F_2| = 2$。记 $M$ 的轨迹为 $C$。
  \begin{enumerate}
    \item 求 $C$ 的方程;
    \item 设点 $T$ 在直线 $x = \frac{1}{2}$ 上,过 $T$ 的两条直线分别交 $C$ 于 $A$,
      $B$ 亮点和 $P$,$Q$ 亮点,且 $|TA| \cdot |TB| = |TP| \cdot |TQ|$,
      求直线 $AB$ 的斜率与直线 $PQ$ 的斜率之和。
  \end{enumerate}
\end{problem}

% 22.
\begin{problem}[points = 12]
  已知函数 $f(x) = x (1 - \ln x)$。
  \begin{enumerate}
    \item 讨论 $f(x)$ 的单调性;
    \item 设 $a$,$b$ 为两个不相等的正数,且 $b \ln a - a \ln b = a - b$,
      证明:$2 < \frac{1}{a} + \frac{1}{b} < \eu$。
  \end{enumerate}
\end{problem}

\begin{solution}
  函数的定义域为 $(0, +\infty)$,
  又 \[f^{\prime}(x) = 1 - \ln x-1 = -\ln x, \score{2}\]
  当 $x \in(0, 1)$ 时, $f^{\prime}(x) > 0$, 当 $x \in(1, +\infty)$ 时, $f^{\prime}(x) < 0$,
  故 $f(x)$ 的递增区间为 $(0,1)$, 递减区间为 $(1, +\infty)$.
\end{solution}



\chapter{2023 年普通高等学校招生全国统一考试}

% \information{
%   姓名\underline{\hspace{6em}},
%   座位号\underline{\hspace{15em}}
% }
% \warning{(在此卷上答题无效)}

\secret

\title{2023 年普通高等学校招生全国统一考试}

\subject{数学(理科)}

\maketitle

本试卷共 4 页,22 题。全卷满分 150 分。考试用时 120 分钟。

\goodluck

\begin{notice}
  \item 答题前,先将自己的姓名、准考证号、考场号、座位号填写在试卷和答题卡上,
    并将准考证号条形码粘贴在答题卡上的指定位置。
  \item 选择题的作答:每小题选出答案后,用 2B 铅笔把答题卡上对应题目的答案标号涂黑。
    写在试卷、草稿纸和答题卡上的非答题区域均无效。
  \item 填空题和解答题的作答:用黑色签字笔直接答在答题卡上对应的答题区域内。
    写在试卷、草稿纸和答题卡上的非答题区域均无效。
  \item 考试结束后,请将本试卷和答题卡一并上交。
\end{notice}



\section{%
  选择题:本题共 8 小题,每小题 5 分,共 40 分。
  在每小题给出的四个选项中,只有一项是符合题目要求的。
}

% 1.
\begin{question}[points = 2]
  设集合 $A = \{x \mid -1 < x < 4\}$,$B = \{2, 3, 4, 5\}$,则 $A \cap B = $ \paren[B]

  \begin{choices}
    \item $\{2\}$
    \item $\{2, 3\}$
    \item $\{3, 4\}$
    \item $\{2, 3, 4\}$
  \end{choices}
\end{question}

% 2.
\begin{question}
  已知 $z = 2 - \iu$,则 $z (\bar{z} + \iu) = $ \paren
  \begin{choices}
    \item $6 - 2\iu$
    \item $2 - 2\iu$
    \item $6 + 2\iu$
    \item $4 + 2\iu$
  \end{choices}
\end{question}

% 3.
\begin{question}
  已知圆锥的底面半径为 $\sqrt{2}$,其侧面展开图为一个半圆,则该圆锥的母线长为 \paren
  \begin{choices}
    \item $2$
    \item $2 \sqrt{2}$
    \item $4$
    \item $4 \sqrt{2}$
  \end{choices}
\end{question}

% 4.
\begin{question}
  下列区间中,函数 $f(x) = 7 \sin \left( x - \frac{\uppi}{6} \right)$ 的单调递增区间是 \paren
  \begin{choices}
    \item $\left( 0               , \frac{\uppi}{2}  \right)$
    \item $\left( \frac{\uppi}{2} , \uppi            \right)$
    \item $\left( \uppi           , \frac{3\uppi}{2} \right)$
    \item $\left( \frac{3\uppi}{2}, 2\uppi           \right)$
  \end{choices}
\end{question}

% 5.
\begin{question}
  已知 $F_1$,$F_2$ 是椭圆 $C \colon \frac{x^2}{9} + \frac{y^2}{4} = 1$ 的两个焦点,
  点 $M$ 在 $C$ 上,则 $|M F_1| \cdot |M F_2|$ 的最大值为 \paren
  \begin{choices}
    \item $13$
    \item $12$
    \item $9$
    \item $6$
  \end{choices}
\end{question}

% 6.
\begin{question}
  若 $\tan\theta = -2$,则 $\frac{\sin\theta (1 + \sin 2\theta)}{\sin\theta + \cos\theta} = $ \paren
  \begin{choices}
    \item $-\frac{6}{5}$
    \item $-\frac{2}{5}$
    \item $\frac{2}{5}$
    \item $\frac{6}{5}$
  \end{choices}
\end{question}

% 7.
\begin{question}
  若过点 $(a, b)$ 可作曲线 $y = \eu^x$ 的两条切线,则 \paren
  \begin{choices}
    \item $\eu^b < a$
    \item $\eu^a < b$
    \item $0 < a < \eu^b$
    \item $0 < b < \eu^a$
  \end{choices}
\end{question}

% 8.
\begin{question}
  有 $6$ 个相同的球,分别标有数字 $1$,$2$,$3$,$4$,$5$,$6$,从中有放回地随机取两次,每次取 $1$ 个球,
  甲表示事件“第一次去出的球的数字是 $1$”,
  乙表示事件“第二次取出的球的数字是 $2$”,
  丙表示事件“两次取出的球的数字之和是 $8$”,
  丁表示事件“两次取出的球的数字之和是 $7$”,则 \paren
  \begin{choices}
    \item 甲与丙相互独立
    \item 甲与丁相互独立
    \item 乙与丙相互独立
    \item 丙与丁相互独立
  \end{choices}
\end{question}



\section{%
  选择题:本题共 4 小题,每小题 5 分,共 20 分。
  在每小题给出的选项中,有多项符合题目要求的。
  全部选对的得 5 分,部分选择的得 2 分,有选错的得 0 分。
}

% 9.
\begin{question}
  有一组样本数据 $x_1, x_2, \dots, x_n$,由 这组数据的到新样本数据 $y_1, y_2, \dots, y_n$,
  其中 $y_i = x_i + c$($i = 1, 2, \dots, n$) 为非零常数,则 \paren
  \begin{choices}
    \item 两组样本数据的样本平均数相同
    \item 两组样本数据的样本中位数相同
    \item 两组样本数据的样本标准差相同
    \item 两组样本数据的样本极差相同
  \end{choices}
\end{question}

% 10.
\begin{question}
  已知 $O$ 为坐标原点,点
  $P_1(\cos\alpha,  \sin\alpha)$,
  $P_2(\cos\beta , -\sin\alpha)$,
  $P_3(\cos(\alpha + \beta),  \sin(\alpha + \beta))$,
  $A(1, 0)$ \paren
  \begin{choices}
    \item $|\overrightarrow{OP_1}| = |\overrightarrow{OP_2}|$
    \item $|\overrightarrow{AP_1}| = |\overrightarrow{AP_2}|$
    \item $\overrightarrow{OA} \cdot \overrightarrow{OP_3}
      = \overrightarrow{OP_1} \cdot \overrightarrow{OP_2}$
    \item $\overrightarrow{OA} \cdot \overrightarrow{OP_1}
      = \overrightarrow{OP_2} \cdot \overrightarrow{OP_3}$
  \end{choices}
\end{question}

% 11.
\begin{question}
  已知点 $P$ 在圆 $(x - 5)^2 + (y - 5)^2 = 16$ 上,点 $A(4, 0)$,$B(0, 2)$,则 \paren
  \begin{choices}
    \item 点 $P$ 到直线 $AB$ 的距离小于 $10$
    \item 点 $P$ 到直线 $AB$ 的距离大于 $2$
    \item 点 $\angle PBA$ 最小时,$|PB| = 3 \sqrt{2}$
    \item 点 $\angle PBA$ 最大时,$|PB| = 3 \sqrt{2}$
  \end{choices}
\end{question}

% 12.
\begin{question}
  在正三棱柱 $ABC$-$A_1 B_1 C_1$ 中,$AB = A A_1 = 1$,点 $P$ 满足
  $\overrightarrow{BP} = \lambda \overrightarrow{BC} + \mu \overrightarrow{BB_1}$,
  其中 $\lambda \in [0, 1]$,$\mu \in [0, 1]$,则 \paren
  \begin{choices}
    \item 当 $\lambda = 1$ 时,$\triangle A B_1 P$ 的周长为定值
    \item 当 $\mu = 1$ 时,三棱锥 $P$-$A_1 B C$ 的体积为定值
    \item 当 $\lambda = \frac{1}{2}$ 时,有且仅有一个点 $P$,使得 $A_1 P \perp BP$
    \item 当 $\mu = \frac{1}{2}$ 时,有且仅有一个点 $P$,使得 $A_1 B \perp \text{平面} A B_1 P$
  \end{choices}
\end{question}



\section{填空题:本题共 4 小题,每小题 5 分,共 20 分。}

% 13.
\begin{question}
  已知函数 $f(x) = x^3 (a \cdot 2^x - 2^{-x})$ 是偶函数,则 $a = $ \fillin[$1$] 。
\end{question}

% 14.
\begin{question}
  已知 $O$ 为坐标原点,抛物线 $C \colon y^2 = 2px$($p > 0$)的焦点为 $F$,
  $P$ 为 $C$ 上一点,$PF$ 与 $x$ 轴垂直,$Q$ 为 $x$ 轴上一点,且 $PQ \perp OP$,
  若 $|FQ| = 6$,则 $C$ 的准线方程为 \fillin[$\dfrac{1}{3}$] 。
\end{question}

% 15.
\begin{question}
  函数 $f(x) = |2x - 1| - 2 \ln x$ 的最小值为 \fillin[width = 4em][] 。
\end{question}

% 16.
\begin{question}
  某校学生在研究民间剪纸艺术时,发现剪纸时经常会沿纸的某条对称轴把纸对折。
  规格为 \qtyproduct{20 x 12}{dm} 的长方形纸,对折 $1$ 次共可以得到
  \qtyproduct{10 x 12}{dm}, \qtyproduct{20 x 6}{dm} 两种规格的图形,
  它们的面积之和 $S_1 = \qty{240}{dm^2}$,
  对折 $2$ 次共可以得到 \qtyproduct{5 x 12}{dm},\qtyproduct{10 x 6}{dm},
  \qtyproduct{20 x 3}{dm} 三种规格的图形,它们的面积之和 $S_2 = \qty{240}{dm^2}$,
  以此类推。则对折 $4$ 次共可以得到不同规格图形的种数为 \fillin ;
  如果对折 $n$ 次,那么 $\sum_{k=1}^n S_k = $ \fillin \unit{dm^2}。
\end{question}



\section{解答题:本题共 6 小题,共 70 分。解答应写出文字说明、证明过程或者演算步骤。}

% 17.
\begin{problem}[points = 10]
  已知数列 $\{a_n\}$ 满足 $a_1 = 1$,$a_{n+1} =
    \begin{cases}
      a_n + 1, \quad \text{$n$ 为奇数,} \\
      a_n + 2, \quad \text{$n$ 为偶数。}
    \end{cases}$
  \begin{enumerate}
    \item 记 $b_n = a_{2n}$,写出 $b_1$,$b_2$,并求数列 $\{b_n\}$ 的通项公式;
    \item 求 $\{a_n\}$ 的前 $20$ 项和。
  \end{enumerate}
\end{problem}

% 18.
\begin{problem}[points = 12]
  某学校组织“一带一路”知识竞赛,有 A,B 两类问题。
  每位参加比赛的同学现在两类问题中选择一类并从中随机抽取一个问题回答,
  若回答错误则该同学比赛结束;
  若回答正确则从另一类问题中再随机抽取一个问题回答,无论回答正确与否,该同学比赛结束。
  A 类问题中的每个问题回答正确的 $20$ 分,否则得 $0$ 分;
  B 类问题中的每个问题回答正确的 $80$ 分,否则得 $0$ 分。

  已知小明能正确回答 A 类问题的概率为 $0.8$,能正确回答 B 类问题的概率为 $0.6$,
  且能正确回答问题的概率与回答次序无关。
  \begin{enumerate}
    \item 若小明先回答 A 类问题,记 $X$ 为小明的累计得分,求 $X$ 的分布列;
    \item 为使累计得分的期望最大,小明应选择先回答哪类问题?并说明理由。
  \end{enumerate}
\end{problem}

% 19.
\begin{problem}[points = 12]
  记 $\triangle ABC$ 的内角 $A$,$B$,$C$ 的对边分别为 $a$,$b$,$c$。
  已知 $b^2 = ac$,点 $D$ 在边 $AC$ 上,$BD \sin\angle ABC = a \sin C$。
  \nopagebreak
  \begin{enumerate}
    \item 证明:$BD = b$;
    \item 若 $AD = 2 DC$,求 $\cos\angle ABC$。
  \end{enumerate}
\end{problem}

% 20.
\begin{problem}[points = 12]
  如图,在三棱锥 $A$-$BCD$ 中,$\text{平面} ABD \perp \text{平面} BCD$,
  $AB = AD$,$O$ 为 $BD$ 的重点。
  \begin{enumerate}
    \item 证明:$OA \perp CD$;
    \item 若 $\triangle OCD$ 是变长为 $1$ 的等边三角形,点 $E$ 在棱 $AD$ 上,
      $DE = 2 EA$,且二面角 $E$-$BC$-$D$ 的大小为 \ang{45},
      求三棱锥 $A$-$BCD$ 的体积。
  \end{enumerate}
\end{problem}

% 21.
\begin{problem}[points = 12]
  在平面直角坐标系 $xOy$ 中,已知点 $F_1 (-\sqrt{17}, 0)$,$F_2 (\sqrt{17}, 0)$,
  点 $M$ 满足 $|M F_1| - |M F_2| = 2$。记 $M$ 的轨迹为 $C$。
  \begin{enumerate}
    \item 求 $C$ 的方程;
    \item 设点 $T$ 在直线 $x = \frac{1}{2}$ 上,过 $T$ 的两条直线分别交 $C$ 于 $A$,
      $B$ 亮点和 $P$,$Q$ 亮点,且 $|TA| \cdot |TB| = |TP| \cdot |TQ|$,
      求直线 $AB$ 的斜率与直线 $PQ$ 的斜率之和。
  \end{enumerate}
\end{problem}

% 22.
\begin{problem}[points = 12]
  已知函数 $f(x) = x (1 - \ln x)$。
  \begin{enumerate}
    \item 讨论 $f(x)$ 的单调性;
    \item 设 $a$,$b$ 为两个不相等的正数,且 $b \ln a - a \ln b = a - b$,
      证明:$2 < \frac{1}{a} + \frac{1}{b} < \eu$。
  \end{enumerate}
\end{problem}

\begin{solution}
  函数的定义域为 $(0, +\infty)$,
  又 \[f^{\prime}(x) = 1 - \ln x-1 = -\ln x, \score{2}\]
  当 $x \in(0, 1)$ 时, $f^{\prime}(x) > 0$, 当 $x \in(1, +\infty)$ 时, $f^{\prime}(x) < 0$,
  故 $f(x)$ 的递增区间为 $(0,1)$, 递减区间为 $(1, +\infty)$.
\end{solution}
\end{document}
