\PassOptionsToPackage{shortlabels}{enumitem}
\documentclass{exam-zh}
\geometry{
  showframe
}
\usepackage{text-figure}
\usepackage{wrapstuff}

\examsetup{
  page = {
    size       = a4paper,
  },
  fillin = {
    type        = line,
    show-answer = true,
    no-answer-type=none,
    text-color=red,
  },
  paren = {
    show-answer = true,
    text-color=red,
  },
  solution = {
%    show-answer = true,
    show-solution = true, % 能否跟fillin, paren一样,设置为show-anwser?
    text-color = red,
  },
}


\begin{document}

% 选择题的选项中有图形
\begin{question}
  下面图形为矩形是?\paren[AB]
  \begin{choices}
    \item {\raisebox{-8mm}{\includegraphics[width=0.15\columnwidth]{example-image.png}}}
    \item {\raisebox{-8mm}{\includegraphics[width=0.15\columnwidth]{example-image.png}}}
    \item {\raisebox{-8mm}{\includegraphics[width=0.15\columnwidth]{example-image.png}}}
    \item {\raisebox{-8mm}{\includegraphics[width=0.15\columnwidth]{example-image.png}}}
  \end{choices}
\end{question}


% 选择题,题干中含有图形,如何让选项和图形实现绕排?
% 该题是一种利用wrapstuff实现的方式,选项实现没有很好的绕排
\begin{question}
  % \begin{wrapstuff}[r,top=0,type=figure,width=0.6\columnwidth]
    % \centering
    % \includegraphics[width=3cm]{example-image.png}
    % \caption{第2题图}
  % \end{wrapstuff}
  如何让各选项参与绕排?\paren[C]
  % \wrapstuffclear
  \textfigure[fig-pos=right,text-ratio=0.9]{
    \begin{choices}
      \item 选项A较长,选项A较长,选项A较长,选项A较长,选项A较长,选项A较长
      \item 选项B较长,选项B较长,选项B较长,选项B较长,选项B较长,选项B较长
      \item 选项C较长,选项C较长,选项C较长,选项C较长,选项C较长,选项C较长
      \item 选项D较长,选项D较长,选项D较长,选项D较长,选项D较长,选项D较长
    \end{choices}
  }{
    \begin{tabular}{c}
      \includegraphics[width=3cm]{example-image.png} \\
      第2题图
    \end{tabular}
  }
\end{question}

\begin{question}
  如何让各选项参与绕排?\paren[C]
  \begin{wrapstuff}[r,type=figure,width=0.6\columnwidth]
    \centering
    \includegraphics[width=3cm]{example-image.png}
    \caption{第2题图}
  \end{wrapstuff}
  % \wrapstuffclear
  \begin{enumerate}[A.]
    \item 选项A较长,选项A较长,选项A较长,选项A较长,选项A较长,选项A较长
    \item 选项B较长,选项B较长,选项B较长,选项B较长,选项B较长,选项B较长
    \item 选项C较长,选项C较长,选项C较长,选项C较长,选项C较长,选项C较长
    \item 选项D较长,选项D较长,选项D较长,选项D较长,选项D较长,选项D较长
  \end{enumerate}
\end{question}


% 选择题,题干中含有图形,如何让选项和图形实现绕排?
% 该题是一种利用text-figure实现的方式,选项C和D没有很好的绕排
\begin{question}
  \textfigure[text-ratio=0.7,vsep=0.5\baselineskip,figure-vsep=-\baselineskip]{
    如何让题干中的图形可以选择起始行数(类似于wrapstuff宏包中wrapstuff环境的参数top的值)?如何让图形和选项实现绕排?\paren[D]
    \begin{choices}
      \item 选项A较长,选项A较长,选项A较长,选项A较长,选项A较长,选项A较长
      \item 选项B较长,选项B较长,选项B较长,选项B较长,选项B较长,选项B较长
      \item 选项C较长,选项C较长,选项C较长,选项C较长,选项C较长,选项C较长
      \item 选项D较长,选项D较长,选项D较长,选项D较长,选项D较长,选项D较长
    \end{choices}
    }{
      \centering
      \includegraphics[width=0.6\columnwidth]{example-image.png}
    }
\end{question}


% 填空题,题干中含有图形,如何让该题的图形和其它题实现绕排?
\begin{wrapstuff}[r,top=1]
  \includegraphics[width=0.3\columnwidth]{example-image.png}
\end{wrapstuff}
\begin{question}
  本题有插图,可以利用wrapstuff实现绕排效果。本题有插图,可以利用wrapstuff实现绕排效果。\fillin[wrapstuff实现的效果较好]。
\end{question}


% 中间的题目会自动实现绕排
\begin{question}
  中间的题目会自动实现绕排。中间的题目会自动实现绕排。中间的题目会自动实现绕排。中间的题目会自动实现绕排。中间的题目会自动实现绕排。中间的题目会自动实现绕排。中间的题目会自动实现绕排。中间的题目会自动实现绕排。中间的题目会自动实现绕排。中间的题目会自动实现绕排。\fillin[中间的题目会自动实现绕排]。
\end{question}


\begin{question}
  可以在某一题后设置wrapstuffclear,清除wrapstuff对后续文本的影响。可以在某一题后设置wrapstuffclear,清除wrapstuff对后续文本的影响。可以在某一题后设置wrapstuffclear,清除wrapstuff对后续文本的影响。\fillin[wrapstuffclear清除影响]。
\end{question}
\wrapstuffclear

% 计算题或问答题,如何实现题干中的图形和答案自动绕排,?
\begin{question}
    带有图形的计算题或问答题。带有图形的计算题或问答题。带有图形的计算题或问答题。带有图形的计算题或问答题。带有图形的计算题或问答题。带有图形的计算题或问答题。


\end{question}
  \begin{flushright}
	\includegraphics[width=0.4\columnwidth]{example-image.png}
  \end{flushright}
\begin{solution}
    这是答案,如何实现答案与题干中的图形绕排?\score{4}

    这是答案,如何实现答案与题干中的图形绕排?\score{2}

    这是答案,如何实现答案与题干中的图形绕排?\score{2}

    这是答案,如何实现答案与题干中的图形绕排?\score{2}
\end{solution}



\end{document}